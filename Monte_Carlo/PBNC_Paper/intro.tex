% !TEX root = PBNC_Paper.tex

%	Statements about the field of research to provide the reader with a setting or context for the problem to be investigated and to claim its centrality or importance.

The area of uncertainty analysis is well established in science and engineering \cite{Morgan}. 
Its role is to provide insight on the impact of uncertainties associated with engineering evaluations. This allows for meaningful and defensible decision making.

Generally, there are three types of uncertainty associated with model prediction: Parameter uncertainty, model uncertainty, and completeness uncertainty. 

%	More specific statements about the aspects of the problem already studied by other researchers, laying a foundation of information already known.

Uncertainty is addressed in fire protection engineering literature \cite{Notarianni:SFPE}. 
And treatment of uncertainty analysis is explicitly required in most technical standards related to fire safety at nuclear reactor facilities \cite{NFPA:805, NUREG:6850}. 
Model bias and uncertainty has been quantified in Nuclear Regulatory Commission (NRC) NUREG-1824~\cite{NUREG_1824_Sup_1} for a number of fire models which are commonly used in nuclear power plant applications. This information can be used as part of a specific methodology to evaluate model uncertainty  that is presented in NRC NUREG-1934~\cite{NUREG_1934}. 

%	Statements that indicated the need for more investigation, creating a gap or research niche for the present study to fill.

The practice of uncertainty analysis in fire modelling is evolving and the methodology employed currently is left to the practitioners' discretion. In many cases, a bounding analysis approach is taken, based on the invalid assumption that uncertainty is addressed by specifying conservative values for input parameters. Bounding analysis by adopting a series of conservative assumptions as a substitute for uncertainty analysis could result in overly conservative decisions (at best) or provide a false understanding of the actual margin of safety.

Although treatment of fire model uncertainty and input parameter sensitivity is clearly addressed in NRC NUREG-1934, parameter uncertainty is not specifically addressed. Given that greatest uncertainty associated with fire protection engineering calculations stems from input parameters, a clear methodology would be useful for practitioners.

%	Statements giving the purpose/objectives of the writer's study or outlining its main activity or findings.

In this paper, we demonstrate the calculation of three cases: 1) the effect of model bias and uncertainty, 2) the effect of input parameter uncertainty, and 3) the combined effect of model bias/uncertainty and input parameter uncertainty.

%	Optional statements that give a positive value or justification for carrying out that study.

This paper provides a straight-forward and practical example of uncertainty analysis for a zone fire modelling application. 