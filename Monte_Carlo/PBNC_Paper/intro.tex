% !TEX root = PBNC_Paper.tex

The area of uncertainty analysis is well established in science and engineering \cite{Morgan}.  Its role is to provide insight on the impact of uncertainties associated with evaluations, which allows for meaningful and defensible decision making in risk assessment.

The practice of uncertainty analysis in fire modelling is evolving and the methodology employed is currently left to the practitioners' discretion. Often, a bounding analysis approach is taken, based on the assumption that uncertainty is addressed by specifying conservative values for input parameters. In some cases, bounding analysis by adopting a series of conservative assumptions as a substitute for uncertainty analysis could result in overly conservative decisions or provide a false understanding of the actual margin of safety. 

Uncertainty is addressed in fire protection engineering literature \cite{Notarianni:SFPE}.  And treatment of uncertainty analysis is explicitly required in most technical standards related to fire safety at nuclear power generating stations \cite{NFPA:805, NUREG_6850}.  Model bias and uncertainty has been quantified in Nuclear Regulatory Commission (NRC) NUREG-1824~\cite{NUREG_1824_Sup_1} for a number of fire models which are commonly used in nuclear power plant applications. This information can be used as part of a specific methodology to evaluate model uncertainty  that is presented in NRC NUREG-1934~\cite{NUREG_1934}. 

The treatment of  uncertainty and sensitivity in NUREG-1934 includes a summary of a derivation for quantifying model uncertainty \cite{McGrattan2011a}, as well as specific calculation examples in Section 4.3. Parameter uncertainty and methods to deal with this kind of uncertainty are discussed separately in Section 4.4; a simple brute force method is shown in  a worked example that propagates a HRR distribution through an algebraic model for predicted flame height, and determines the probability of flames reaching a certain height.

Given that input parameter uncertainty and model uncertainty are a both addressed separately in NUREG-1934; an apparent logical extension of the methodologies presented would be to treat both kinds of uncertainty in a manner that shows the combined effect of both kinds of uncertainty.  It is the aim of this paper to provide practitioners with a straight-forward and practical example of combined model and input uncertainty analysis for a zone fire modelling application. 

In this paper, we have selected  a worked example of a fire scenario that is typical for a nuclear power generating station to demonstrate the calculation of three cases: 1) the effect of model bias and uncertainty, 2) the effect of input parameter uncertainty, and 3) the combined effect of model bias/uncertainty and input parameter uncertainty. 

